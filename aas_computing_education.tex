\documentclass[11pt]{article}

\usepackage{mathpazo}

\usepackage[margin=1in]{geometry}

\usepackage{hyperref}

\usepackage[defaultsans]{cantarell}
\usepackage[T1]{fontenc}

\usepackage[para]{footmisc}

\usepackage[small,compact]{titlesec}
\titlespacing{\subsection}{0pt}{*2}{*0.125}
\titleformat*{\subsection}{\sffamily\bfseries}
\titleformat*{\section}{\sffamily\bfseries}

\usepackage{fancyhdr}
\pagestyle{fancy}
\rhead{\footnotesize \sf The Importance of Computation in Astronomy Education}
\lhead{}

\newenvironment{squishlist}                                                     
  {\begin{itemize}                                                              
    % set spacing between items                                                 
    \addtolength{\itemsep}{-0.33\baselineskip}                                  
    % set spacing between lines                                                 
    %\addtolength{\baselineskip}{-0.25\baselineskip}                            
   }                                                                            
  { \end{itemize} }                                                             
                       

\usepackage{tcolorbox}

\begin{document}

\thispagestyle{plain}

\begin{center}
{\Large \sffamily \bfseries The Importance of Computation in Astronomy Education} \\
{
M. Zingale\footnote{Stony Brook University}
F.X.~Timmes\footnote{Arizona State University}
R.~Fisher\footnote{U Mass Dartmouth}
}
\end{center}

\begin{tcolorbox}
{\sffamily \bfseries Executive Summary:} Computational skills are required
across all disciplines, theory and observation inclusive, in astronomy.
Many students enter graduate programs without sufficient skills
to solve computational problems in their core classes, and to jump in and
contribute right away in research. We recommend a push for computational
literacy in the early undergraduate years, and familiarity with fundamental
software carpentry skills as well as core numerical methods by the
completion of an undergraduate degree in Astronomy.  Further, we recommend
that these skills be built on throughout graduate education by including
computational problems in the core astronomy classes, and discuss
the role that the AAS can take.  Finally, we discuss
the role of open source software in astronomy education.

\vspace {0.1in} We recommend the AAS Education Task Force advocate for
a significant increase in computational literacy at the undergraduate
and graduate levels.

\vspace {0.1in} We recommend the AAS modestly fund efforts aimed at
providing Open Education Resources (OER) that will significantly
impact computational literacy at the undergraduate and graduate
levels.


\end{tcolorbox}

\section{Computational Needs in Astronomy \& Astrophysics}

Computational skills are required at all levels of research in
astronomy.  Theory is dominated by simulation codes, e.g. stellar
evolution or multidimensional, multiphysics hydrodynamics codes,
written in a variety of languages (C++, Fortran, and python being the
most popular).  Observational astronomy is entirely digital, and the
workflow takes the form of a software pipeline to reduce and analyze
data.  IDL was a popular player here a decade ago, but is rapidly
being replaced by python, and core libraries such as
AstroPy\footnote{\url{http://www.astropy.org/}} are being developed to
fill this need.

Furthermore, the workflow in Astronomy is often expressed in terms of
a UNIX-like environment with OS X or Linux serving as the OS of
choice.  Students coming out of high school may be unfamiliar with
(and put off from) the command line and the power it enables.

\section{Undergraduate Education}

Many undergraduate programs encourage physics and astronomy majors to
take some computer programming classes.  For example, the State
University of NY (SUNY) transfer path for physics requires intro to CS
in the first 2
years~\footnote{\url{http://www.suny.edu/attend/get-started/transfer-students/suny-transfer-paths/pdf/transferSUNY_Physics.pdf}}.
Ideally, students should also learn elements of scientific computing and
basic numerical analysis.  Many other institutions have such
requirements, but it is not universal.  We recommend that the AAS
advocate making scientific computation literacy a requirement for
undergraduate majors.

What is also not universally done is to follow through by
applying introductory computing skills to problems in upper-level astronomy
courses.  There are many opportunities for this.  In stellar
structure, Open Source tools developed by the community can be used to
understand all stages of stellar evolution. Open data archives enable
access to galactic and extragalactic data relevant to all of these
core topics, and present educators with outstanding opportunities to
bring students directly into contact with real-world data, and to
integrate data analysis and computation into the standard
undergraduate curriculum. Some examples of data-driven exercises
include :
\begin {squishlist}

\item Inferring the mass, radius, and density of the historic transiting exoplanet HD209458b

\item Creation of a HR diagram from Tycho data

\item Calculation of stellar interiors using MESA-web

\item Determination of the value of the Hubble constant $H_0$ from Type Ia light curve data

\item Analysis of the gravitational waves from the historic binary black hole merger GW150914

\end {squishlist}

Many astrophysical hydrodynamics codes are freely available and
provide sample problems that can be used as the basis for lesson
plans (for instance, shocks in the ISM).

Instructors can guide students with highly-structured homework
exercises and class projects which empower students to become
acquainted with the power of computation on realistic ``real world''
examples, and learn foundational data analysis skills.

The AAS should encourage the development and sharing of computational
tools, lessons, and projects amongst their members. Broad
dissemination of software carpentry lesson training plans, exemplified
by recent AAS workshops on this topic, will help familiarize students
with these fundamental software skills, which also are transferable to
a wide range of careers outside of astronomy.





\section{Graduate Education}

A common issue at the graduate level is that specialty classes (e.g.,
one focusing on computational hydrodynamics) tend to attract only a
small number of students, making it difficult to justify their regular
offering.

At the graduate level, a popular way to train students in the
specialized codes and techniques used in each sub-discipline are summer
schools.  AAS should encourage summer schools and training sessions,
perhaps in association with annual meetings.  The Software
Carpentry\footnote{\url{http://software-carpentry.org/}} sessions at
recent meetings are an excellent example of this training, and there is a lot of
potential further expand upon these sessions.  Community code projects should be
encouraged by the AAS to offer similar tutorial sessions for their domain
codes.

Additionally, as Software Carpentry has proven to be such a valuable
training method in scientific computing, AAS meetings should also
offer instructor training sessions for Software Carpentry, so members
can offer workshops at their own institutions.


\section{Open Source and Open Education Resources}

Open educational resources (OER) are freely accessible, openly
licensed documents and media that are useful for teaching, learning,
assessing, and research. OER are among the leading trends in distance
education, yet there is a paucity of quality OER materials for
astronomy. A few notables include: (1)
\href{http://astroedu.iau.org}{http://astroedu.iau.org} is targeted
at K-12. It was launched in February 2015 and supported by the IAU
Office for Astronomy Development; (2) The resources available on the
Astrobetter
Wiki\footnote{\url{http://www.astrobetter.com/wiki/Wiki+Home}} which
include links to user-contributed class slides, animations, texts,
etc. (3) open-licensed texts for undergraduate and graduate education,
such as those hosted on the Open Astrophysics
Bookshelf\footnote{\url{https://open-astrophysics-bookshelf.github.io}}
and other
sites\footnote{\url{http://www.pa.msu.edu/~ebrown/lecture-notes.html}}
(4) MESA-Web\footnote{\url{http://mesa-web.asu.edu}}, a
web-based portal to the stellar evolution instrument, Modules for
Experiments in Stellar Astrophysics ({\tt MESA}) and aimed at
secondary and undergraduate education. To date, this unfunded OER
effort has run over 1200 stellar evolution models since March 2015
from jobs submitted by 62 institutions in the USA.

Finally, the advent of IPython/Jupyter notebooks has made the
deployment of interactive computation-based exercises much easier.  We
recommend that the AAS recognize the value of OER and explore ways to
fund these efforts in our community.

\section{Careers}

Finally, many Astronomy PhDs will not stay in academia, since each
faculty member will produce many PhDs during their academic career.
As a result, graduates will find jobs in industry and at national
labs. Computational skills are perhaps among the most transferable
skills which an astronomy graduate student will acquire in their
education and training, and can make a grad very attractive to a wide
range of potential employers.  Advocacy by the AAS for increased
computational literacy in all stages of education will greatly help
our graduates.


\end{document}
