\documentclass[11pt]{article}

\usepackage{mathpazo}

\usepackage[margin=1in]{geometry}

\usepackage{hyperref}

\usepackage[defaultsans]{cantarell}
\usepackage[T1]{fontenc}


\usepackage[small,compact]{titlesec}
\titlespacing{\subsection}{0pt}{*2}{*0.125}
\titleformat*{\subsection}{\sffamily\bfseries}
\titleformat*{\section}{\sffamily\bfseries}

\usepackage{fancyhdr}
\pagestyle{fancy}
\rhead{\footnotesize \sf The Importance of Computation in Astronomy Education}
\lhead{}

\usepackage{tcolorbox}

\begin{document}

\thispagestyle{plain}

\begin{center}
{\Large \sffamily \bfseries The Importance of Computation in Astronomy Education} \\
{M. Zingale\footnote{Stony Brook University}}
\end{center}

\begin{tcolorbox}
{\sffamily \bfseries Executive Summary:} Computational skills are required
across all disciplines, theory and observation inclusive, in astronomy.
Many students enter graduate programs without sufficient skills
to solve computational problems in their core classes and to jump in and
contribute right away in research.  We recommend a push for computational
literacy in the early undergraduate years and core numerical methods by
completion of an undergraduate degree in Astronomy.  Further, we recommend
that these skills be built on throughout graduate education by including
computational problems in the core astronomy classes, and discuss
the role that the AAS can take.  Finally, we discuss
the role of open source software in astronomy education.

\vspace {0.1in} We recommend the AAS Education Task Force advocate for
a significant increase in computational literacy at the undergraduate
and graduate levels.

\vspace {0.1in} We recommend the AAS modestly fund efforts aimed at
providing Open Education Resources (OER) that will significantly
impact computational literacy at the undergraduate and graduate
levels.


\end{tcolorbox}

\section{Computational Needs in Astronomy \& Astrophysics}

Computational skills are required at all levels of research in
astronomy.  Theory is dominated by simulation codes, e.g. stellar
evolution or multidimensional, multiphysics hydrodynamics codes,
written in a variety of languages (C++, Fortran, and python being the
most popular).  Observational astronomy is entirely digital, and the
workflow takes the form of a software pipeline to reduce and analyze
data.  IDL was a popular player here a decade ago, but is rapidly
being replaced by python, and core libraries such as AstroPy are being
developed to fill this need.

Furthermore, the workflow in Astronomy is often expressed in terms of
a UNIX-like environment with OS X or Linux serving as the OS of
choice.  Students coming out of high school may be unfamaliar with
(and put off from) the commandline and the power it enables.

\section{Undergraduate Education}

Many undergraduate programs encourage physics and astronomy majors to
take some for of computer programming class.  For example, the State
University of NY (SUNY) transfer path for physics requires intro to CS
in first 2
years~\footnote{\url{http://www.suny.edu/attend/get-started/transfer-students/suny-transfer-paths/pdf/transferSUNY_Physics.pdf}}.
Ideally, students would learn elements of scientific computing
and basic numerical analysis.

What is not universally done is following up on this introduction
by applying the computing skills to problems in upper-level
astronomy courses.  There are lots of opportunities for
this, for example in stellar structure, using Open Source
tools developed by the community.

AAS should encourage the development and sharing of computational
tools, lessons, and projects amongst their members.



\section{Graduate Education}

A common issue at the graduate level is that speciality classes (e.g.,
one focusing on computational hydrodynamics) tend to attract only a
small number of students, making it difficult to justify their regular
offering.

At the graduate level, a popular way to train students in the specialized
codes and techniques used in each subdiscipline are summer schools.
AAS should encourage summer schools and training sessions, perhaps in 
association with annual meetings.  The Software Carpentry sessions
at recent meetings is an excellent example of this, and there is 
a lot of potential for this to be expanded.  Community code projects
should be encouraged by AAS to offer similar tutorial sessions for 
their domain codes.  

Additonally, as Software Carpentry has proven to be such a valuable
training method in scientific computing, AAS meetings should also
offer instructor training sessions for Software Carpentry, so members
can offer workshops at their own institutions.


\section{Open Source and Open Education Resources}

\href{https://open-astrophysics-bookshelf.github.io}{https://open-astrophysics-bookshelf.github.io}
is a collection of OER texts on various
topics in astrophysics. You are free to use them as-is, make mash-ups
of different texts, or contribute back to their development.

\href{http://www.pa.msu.edu/~ebrown/lecture-notes.html}{http://www.pa.msu.edu/~ebrown/lecture-notes.html}
is a collection of OER texts developed by Ed Brown, Michigan State.


Stellar evolution instruments can be complicated to configure and
(properly), so Carl Fields and Frank Timmes developed {\tt MESA}-Web,
a web-based portal to the stellar evolution instrument, Modules for
Experiments in Stellar Astrophysics ({\tt MESA}).  {\tt MESA}-Web can
be used for education purposes to calculate stellar models over a
range of physical parameters.  To date, this unfunded OER effort has
run XXX stellar evolution models since YYY, encompassing N institions
in the USA and M institutions worldwide.



\subsection{Python and Jupyter Notebooks}

Notebooks have found wide adoption in the classroom and allow for
interactive in-class activities and for the student to reply the
lecture on their own outside of class.

\subsection{Simulation Codes}




\section{Careers}

Many Astronomy PhDs will not stay in academia, since each faculty member
will produce many PhDs during their academic career.  As a result, graduates
will find jobs in industry and at national labs.  Computational skills 
can make a grad very attractive to potential employers.


\end{document}
