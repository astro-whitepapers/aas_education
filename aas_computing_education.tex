\documentclass[11pt]{article}

\usepackage{mathpazo}

\usepackage[margin=1in]{geometry}

\usepackage{hyperref}

\usepackage[defaultsans]{cantarell}
\usepackage[T1]{fontenc}


\usepackage[small,compact]{titlesec}
\titlespacing{\subsection}{0pt}{*2}{*0.125}
\titleformat*{\subsection}{\sffamily\bfseries}
\titleformat*{\section}{\sffamily\bfseries}

\usepackage{fancyhdr}
\pagestyle{fancy}
\rhead{\footnotesize \sf The Importance of Computation in Astronomy Education}
\lhead{}

\usepackage{tcolorbox}

\begin{document}

\thispagestyle{plain}

\begin{center}
{\Large \sffamily \bfseries The Importance of Computation in Astronomy Education} \\
{M. Zingale\footnote{Stony Brook University}}
\end{center}

\begin{tcolorbox}
{\sffamily \bfseries Executive Summary:} Computational skills are required
across all disciplines, theory and observation inclusive, in astronomy.
Many students enter graduate programs without sufficient skills
to solve computational problems in their core classes and to jump in and
contribute right away in research.  We recommend a push for computational
literacy in the early undergraduate years and core numerical methods by
completion of an undergraduate degree in Astronomy.  Further, we recommend
that these skills be built on throughout graduate education by including
computational problems in the core astronomy classes.  Finally, we discuss
the role of open source software in astronomy education.
\end{tcolorbox}

\section{Computational Needs in Astronomy \& Astrophysics}

\section{Undergraduate Preparation}

\section{Graduate Preparation}

\section{Open Source and Education}






\end{document}
