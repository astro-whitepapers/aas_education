\documentclass[11pt]{article}

\usepackage{mathpazo}

\usepackage[margin=1in]{geometry}

\usepackage{hyperref}

\usepackage[defaultsans]{cantarell}
\usepackage[T1]{fontenc}

\usepackage[para]{footmisc}

\usepackage[small,compact]{titlesec}
\titlespacing{\subsection}{0pt}{*2}{*0.125}
\titleformat*{\subsection}{\sffamily\bfseries}
\titleformat*{\section}{\sffamily\bfseries}

\usepackage{fancyhdr}
\pagestyle{fancy}
\rhead{\footnotesize \sf The Importance of Computation in Astronomy Education}
\lhead{}

\newenvironment{squishlist}                                                     
  {\begin{itemize}                                                              
    % set spacing between items                                                 
    \addtolength{\itemsep}{-0.33\baselineskip}                                  
    % set spacing between lines                                                 
    %\addtolength{\baselineskip}{-0.25\baselineskip}                            
   }                                                                            
  { \end{itemize} }                                                             
                       

\usepackage{tcolorbox}

\begin{document}

\thispagestyle{plain}

\mbox{ }\vspace{-0.7in}

\begin{center}
{\Large \sffamily \bfseries The Importance of Computation in Astronomy Education} \\
{
M. Zingale\footnote{Stony Brook University}
F.X.~Timmes\footnote{Arizona State University}
R.~Fisher\footnote{U Mass Dartmouth}
}
\end{center}

\begin{tcolorbox}
{\sffamily \bfseries Executive Summary:} Computational skills are required
across all astronomy disciples. 
Many students enter degree programs without sufficient skills
to solve computational problems in their core classes or contribute right away to research. 
We recommend advocacy for computational literacy, familiarity with fundamental
software carpentry skills, and basic numerical methods by the
completion of an undergraduate degree in Astronomy.  

\vspace {0.1in} We recommend the AAS Education Task Force advocate for
a significant increase in computational literacy.

\vspace {0.1in} We encourage the AAS to modestly fund efforts aimed at
providing Open Education Resources (OER) that will significantly
impact computational literacy in astronomy education.


%{\sffamily \bfseries Executive Summary:} Computational skills are required
%across all disciplines, theory and observation inclusive, in astronomy.
%Many students enter graduate programs without sufficient skills
%to solve computational problems in their core classes, and to jump in and
%contribute right away in research. We recommend a push for computational
%literacy in the early undergraduate years, and familiarity with fundamental
%software carpentry skills as well as core numerical methods by the
%completion of an undergraduate degree in Astronomy.  Further, we recommend
%that these skills be built on throughout graduate education by including
%computational problems in the core astronomy classes, and discuss
%the role that the AAS can take.  Finally, we discuss
%the role of open source software in astronomy education.

%\vspace {0.1in} We recommend the AAS Education Task Force advocate for
%a significant increase in computational literacy in undergraduat and graduate education.

%\vspace {0.1in} We recommend the AAS modestly fund efforts aimed at
%providing Open Education Resources (OER) that will significantly
%impact computational literacy at the undergraduate and graduate
%levels.


\end{tcolorbox}

\section{Computational Needs in Astronomy \& Astrophysics}

Computational skills are required at all levels of education
and research in astronomy.  Theory is dominated by simulation instruments, 
often written in compiled and interpreted languages. 
Observational astronomy is entirely digital, with 
a software pipeline reducing and analyzing the data.  Community tools, such as the python-based
AstroPy\footnote{\url{http://www.astropy.org/}}, are actively developed for the pipelines.
The workflow in astronomy is often expressed in UNIX-like environments such as  OS X or Linux.
Students in secondary or undergraduate programs may be unfamiliar with
(and put off from) the command line and the job skills it enables.

\section{Undergraduate Education}

Many astronomy and astrophysics programs encourage their majors to
take some computer programming classes.  For example, the State
University of New York transfer path for physics requires an Introduction to Computer Science 
in the first 2
years~\footnote{\url{http://www.suny.edu/attend/get-started/transfer-students/suny-transfer-paths/pdf/transferSUNY_Physics.pdf}\hfill}.
However, this is where the encouragement of developing essential, transferrable job skills frequently ends. 

Astronomy students should be versed in elements of scientific computing and
basic numerical analysis in astronomy courses that leverage
community developed infrastructure. For example, Open Source web-based tools can 
be used to discover all stages of stellar evolution. Open data archives enable access to
galactic and extragalactic data.  These, and other, communuty
developments offer educators offer outstanding opportunities to bring
students directly into contact with real-world data, and to integrate
data analysis and computation into the curriculum. Some examples of
data-driven educational exercises include:
\begin {squishlist}
\item Inferring the mass, radius, and density of the historic transiting exoplanet HD209458b
\item Creation of a HR diagram from Tycho data
\item Examination of stellar interiors using MESA-web
\item Determination of the Hubble constant $H_0$ from Supernova Type Ia light curve data
\item Analysis of gravitational waves from the historic binary black hole merger GW150914
\end {squishlist}

We applaud the AAS's advocacy for increased scientific computation
literacy as exemplified by the Hack Day at recent AAS meetings.  We
suggest the AAS enhance its encouragement of sharing computational
tools, educational lessons, and projects amongst their members.


%Broad
%dissemination of software carpentry lesson training plans, exemplified
%by recent AAS workshops on this topic, will help familiarize students
%with these fundamental software skills, which also are transferable to
%a wide range of careers outside of astronomy.
%Many astrophysical hydrodynamics codes are freely available and
%provide sample problems that can be used as the basis for lesson
%plans (for instance, shocks in the ISM).
%
%Instructors can guide students with highly-structured homework
%exercises and class projects which empower students to become
%acquainted with the power of computation on realistic ``real world''
%examples, and learn foundational data analysis skills.





\section{Graduate Education}

A popular way to train graduate students in specialized codes and
techniques used in each astronomy discipline are summer schools
and workshops. We encourage the AAS to 

\begin{squishlist}

\item extend and promote these training sessions
in association with the AAS meetings.  
For example, the Software Carpentry\footnote{\url{http://software-carpentry.org/}} 
sessions at recent AAS meetings are an excellent example of this training, 
and there is significant potential further to expand upon these sessions.  

\item offer software instrument specific training sessions
at the AAS meetings. For example, ``Best practices for CLOUDY\footnote{\url{http://trac.nublado.org}}''
or ``Advanced MAESTRO\footnote{\url{ https://ccse.lbl.gov/Research/MAESTRO}}'' can
provide critical community networking opportunities for graduate students.

\item orgranize Instructor Training sessions for Software Carpentry,
to allow participants to offer workshops at their own institutions.

\end{squishlist}

%A common issue at the graduate level is that specialty classes (e.g.,
%one focusing on computational hydrodynamics) tend to attract only a
%small number of students, making it difficult to justify their regular
%offering.

%At the graduate level, a popular way to train students in the
%specialized codes and techniques used in each sub-discipline are summer
%schools.  AAS should encourage summer schools and training sessions,
%perhaps in association with annual meetings.  The Software
%Carpentry\footnote{\url{http://software-carpentry.org/}} sessions at
%recent meetings are an excellent example of this training, and there is a lot of
%potential further expand upon these sessions.  Community code projects should be
%encouraged by the AAS to offer similar tutorial sessions for their domain
%codes.

%Additionally, as Software Carpentry has proven to be such a valuable
%training method in scientific computing, AAS meetings should also
%offer instructor training sessions for Software Carpentry, so members
%can offer workshops at their own institutions.


\section{Open Source and Open Education Resources}

OER are freely accessible, openly
licensed documents and media for teaching, learning,
assessing, and research. OER are among the leading trends in 
education, yet there is a paucity of quality material for
astronomy. A few notable exceptions include: (1)
astroEDU\footnote{\url{http://astroedu.iau.org}} which
launched in February 2015, targets K-12, and is supported by the IAU
Office for Astronomy Development; (2) Astrobetter
Wiki\footnote{\url{http://www.astrobetter.com/wiki/Wiki+Home}} which
include links to user-contributed class slides, animations, texts;
(3) open-licensed texts such as Open Astrophysics Bookshelf\footnote{\url{https://open-astrophysics-bookshelf.github.io}}
and others\footnote{\url{http://www.pa.msu.edu/~ebrown/lecture-notes.html}};
(4) MESA-Web\footnote{\url{http://mesa-web.asu.edu}}, a
web-based portal for stellar evolution aimed at secondary and undergraduate education;
(5) IPython/Jupyter notebooks for deployment of interactive computation-based exercises.

\begin{squishlist}
\item
We encourage the AAS to modestly fund efforts aimed at providing OER
material that will significantly impact computation in astronomy education.
\end{squishlist}

\section{Careers}

Computational skills and critical thinking are among the most
transferable job skills that an astronomy education can provide.
We encourage the continued advocacy by the AAS for increased
computational literacy, exemplified by the Hack Day at recent AAS meetings, 
to provide skills that employers consistently seek.


%Finally, many Astronomy PhDs will not stay in academia, since each
%faculty member will produce many PhDs during their academic career.
%As a result, graduates will find jobs in industry and at national
%labs. Computational skills are perhaps among the most transferable
%skills which an astronomy graduate student will acquire in their
%education and training, and can make a grad very attractive to a wide
%range of potential employers.  Advocacy by the AAS for increased
%computational literacy in all stages of education will greatly help
%our graduates.


\end{document}
