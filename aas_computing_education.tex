\documentclass[11pt]{article}

\usepackage{mathpazo}

\usepackage[margin=1in]{geometry}

\usepackage{hyperref}

\usepackage[defaultsans]{cantarell}
\usepackage[T1]{fontenc}


\usepackage[small,compact]{titlesec}
\titlespacing{\subsection}{0pt}{*2}{*0.125}
\titleformat*{\subsection}{\sffamily\bfseries}
\titleformat*{\section}{\sffamily\bfseries}

\usepackage{fancyhdr}
\pagestyle{fancy}
\rhead{\footnotesize \sf The Importance of Computation in Astronomy Education}
\lhead{}

\usepackage{tcolorbox}

\begin{document}

\thispagestyle{plain}

\begin{center}
{\Large \sffamily \bfseries The Importance of Computation in Astronomy Education} \\
{M. Zingale\footnote{Stony Brook University}}
\end{center}

\begin{tcolorbox}
{\sffamily \bfseries Executive Summary:} Computational skills are required
across all disciplines, theory and observation inclusive, in astronomy.
Many students enter graduate programs without sufficient skills
to solve computational problems in their core classes and to jump in and
contribute right away in research.  We recommend a push for computational
literacy in the early undergraduate years and core numerical methods by
completion of an undergraduate degree in Astronomy.  Further, we recommend
that these skills be built on throughout graduate education by including
computational problems in the core astronomy classes.  Finally, we discuss
the role of open source software in astronomy education.
\end{tcolorbox}

\section{Computational Needs in Astronomy \& Astrophysics}

Computational skills are required at all levels of research in
astronomy.  Theory is dominated by simulation codes, e.g. stellar
evolution or multidimensional, multiphysics hydrodynamics codes,
written in a variety of languages (C++, Fortran, and python being the
most popular).  Observational astronomy is entirely digital, and the
workflow takes the form of a software pipeline to reduce and analyze
data.  IDL was a popular player here a decade ago, but is rapidly
being replaced by python, and core libraries such as AstroPy are being
developed to fill this need.

Furthermore, the workflow in Astronomy is often expressed in terms of
a UNIX-like environment with OS X or Linux serving as the OS of
choice.  Students coming out of high school may be unfamaliar with
(and put off from) the commandline and the power it enables.

\section{Undergraduate Education}

SUNY transfer path for physics requires intro to CS in first 2 years

 SBU majors require intro class



\section{Graduate Education}

At the graduate level, a popular way to train students in the specialized
codes and techniques used in each subdiscipline are summer schools.

A common issue at the graduate level is that speciality classes (e.g.,
one focusing on computational hydrodynamics) tend to attract only a
small number of students, making it difficult to justify their regular
offering.

\subsection{High-Performance Computing}

Summer schools


\section{Open Source and Education}

\subsection{Python and Jupyter Notebooks}

Notebooks have found wide adoption in the classroom and allow for
interactive in-class activities and for the student to reply the
lecture on their own outside of class.

\subsection{Simulation Codes}




\section{Careers}

Many Astronomy PhDs will not stay in academia, since each faculty member
will produce many PhDs during their academic career.  As a result, graduates
will find jobs in industry and at national labs.  Computational skills 
can make a grad very attractive to potential employers.


\end{document}
